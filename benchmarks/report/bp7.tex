
\subsection{BP 7:
 Monai valley beach (Laboratory)}

{\bf Documentation:}  PMEL-135, pp 6 \& 45-46.

\subsubsection{Problems encountered}

\begin{itemize}
\item The input data only goes out to 20 seconds.

\item The first waves are modeled well but later waves are not seen in the
computation.  This is perhaps related to first point.  

\item Data is not provided for run-up in the valley.
\end{itemize}

\subsubsection{What we did}

\begin{itemize}
\item Used $g=9.81$ and no friction.
\item We used the given initial wave to specify a boundary condition at the left
boundary up to time 20.  This was done by filling ghost cells each time step
at the left edge of the computational domain, with cell centers at $x = -
\frac 1 2 \Delta x$ and $x = -\frac 3 2 \Delta x$.
The depth at any such point $x_0$ at arbitrary time $t_0$ should be roughly
the same as the depth at $x=0$ at time $t_0 - x_0/c_0$, where $c_0 =
\sqrt{gh_0}$ is the wave speed based on the constant depth.  This value was
estimated by 
interpolating from the given time trace at $x=0$.
To fully specify the ghost cell values we also need the velocity or momentum
in these cells.  This is estimated by assuming the wave is a right-going
simple wave, so that the Riemann invariant $u - 2\sqrt{gh}$ is constant
through the wave.  Inserting the interpolated $h$ value and then setting this
equal to $-2\sqrt{gh_0}$ gives the expression for the velocity $u$ used in
the ghost cell.

\item After time 20, switched to non-reflecting boundary condition 
at left boundary, so reflected waves exit.  
(Note that the wave tank was much longer than computational domain specified.)
\item We solved on $423\times 243$ grid (same as bathymetry)
\item We also solved on $160\times 100$ grid (coarser grid, 
not aligned with bathymetry) for comparison.
\end{itemize} 

\subsubsection{Gauge comparisons}

See \Fig{bp7gauges}.

\begin{figure}[ht]
\hfil\includegraphics[width=2.8in]{bp7/figs423/gauge0005fig300.png}\hfil
\hfil\includegraphics[width=2.8in]{bp7/figs160/gauge0005fig300.png}\hfil
\vskip 5pt
\hfil\includegraphics[width=2.8in]{bp7/figs423/gauge0007fig300.png}\hfil
\hfil\includegraphics[width=2.8in]{bp7/figs160/gauge0007fig300.png}\hfil
\vskip 5pt
\hfil\includegraphics[width=2.8in]{bp7/figs423/gauge0009fig300.png}\hfil
\hfil\includegraphics[width=2.8in]{bp7/figs160/gauge0009fig300.png}\hfil
\caption{\label{fig:bp7gauges} 
Left column: on $423\times 243$ grid (same as given bathymetry).
Right column: $160\times 100$ grid (coarser and not aligned with
bathymetry).
  }
\end{figure}

\subsubsection{Frame comparisons}

See \Fig{bp7framesA} and \Fig{bp7framesA}.

\begin{figure}[ht]
\hfil\includegraphics[width=1.5in]{bp7/Frame10.png}\hfil
\hfil\includegraphics[width=2.8in]{bp7/figs423/frame0005fig10.png}\hfil
\vskip 5pt
\hfil\includegraphics[width=1.5in]{bp7/Frame25.png}\hfil
\hfil\includegraphics[width=2.8in]{bp7/figs423/frame0007fig10.png}\hfil
\vskip 5pt
\hfil\includegraphics[width=1.5in]{bp7/Frame40.png}\hfil
\hfil\includegraphics[width=2.8in]{bp7/figs423/frame0009fig10.png}\hfil
\caption{\label{fig:bp7framesA} 
Left column: Frames from the movie.
Right column: Zoomed view of computation.
  }
\end{figure}


\begin{figure}[ht]
\hfil\includegraphics[width=1.5in]{bp7/Frame55.png}\hfil
\hfil\includegraphics[width=2.8in]{bp7/figs423/frame0011fig10.png}\hfil
\vskip 5pt
\hfil\includegraphics[width=1.5in]{bp7/Frame70.png}\hfil
\hfil\includegraphics[width=2.8in]{bp7/figs423/frame0013fig10.png}\hfil
\vskip 5pt
\caption{\label{fig:bp7framesB} 
Left column: Frames from the movie.
Right column: Zoomed view of computation.
  }
\end{figure}


\subsection{Lessons learned}

\todo{Make any relevant comments or observation regarding the benchmark itself,
the available data provided, the relevance of the benchmark to tsunami
science in general and to validating your specific model in particular. 
Report problems and make recommendations regarding improving the benchmark
or its data, as appropriate.}

