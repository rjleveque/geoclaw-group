\newsection
\subsection{BP 1: Single wave on a simple beach (Analytic)}

{\bf Documentation:} A description of this benchmark problem is provided by \cite{bp-description} and \cite{SynolakisBernard:pmel135}.

\subsubsection{Problem Description}
The focus is on comparing computed and analytic solutions for a wave incident on a simple beach, in which:
\begin{itemize}
\item The bathymetry consists of a deep region of constant depth $d$
connected to a sloping beach of angle $\beta = \text{arccot}(19.85)$. 
Note that the toe of the beach is located at $x = X_0 = d \text{cot} \beta$
\item The initial waveform of the wave is given by 
\begin{equation}
\eta(x,0) = H \text{sech}^2(\gamma (x - X_1)/d)
\end{equation}
where $L = \text{arccosh}(\sqrt(20))/\gamma$, $X_1 = X_0 + L$, 
and $\gamma = \sqrt(3H/4d)$. The speed of the wave is given by the following: 
\begin{equation}
u(x,0)=-\sqrt{g/d}\eta(x,0)
\end{equation}
\end{itemize}

\begin{figure}[ht]
\hfil\includegraphics[width=5.0in]{bp1/bp1domain}\hfil
\caption{\label{fig:bp1domain} 
Sketch of canonical beach and approaching wave.
 }
\end{figure}

\subsubsection{Tasks}
\begin{itemize}
\todo{Paul: Please reproduce here the task list in Dmitry's description.}

\end{itemize}

\subsubsection{Problems encountered}

\begin{itemize}
\item The analytic solution of the wave equation was hard to determine and compute.  The analytic solution was obtained from the benchmark problem champion; it would be very helpful if it were provided in an Excel file as part of the benchmark problem description.  
\item No analytical solution was provided for time $t = 25s$
\item The Clawpack code does not currently include maximum runup calculations. An additional module had to be written. 
\end{itemize}

\subsubsection{What we did}

\begin{itemize}
\item Used $g=1$ and no friction.
\item The problem was solved on an $800\times 2$ grid, where the x domain spanned x = -10 to 60.
\item Variable time stepping was allowed, based on a CFL number of 0.9
\end{itemize} 

\subsubsection{Results}
\begin{itemize}
\item Task 1. Good agreement between computed and analytic water level profiles at $t = 35(d/g)^{1/2}$, $t = 45(d/g)^{1/2}$, $t = 55(d/g)^{1/2}$, $t = 65(d/g)^{1/2}$ is presented in Figure \Fig{bp1frames}.  Data were missing from file {\tt canonical\_profiles.txt} for $t=25(d/g)^{1/2}$, so this time was omitted.
\item Task 2. Good agreement between computed and analytic water levels at locations $x/d = 0.25$ and $x/d = 9.95$ during the propagation and reflection of the wave is presented in Figure \Fig{bp1gauges}.
\item Task 3. Maximum runup on the beach was 0.085, as presented in the time series of runup values in Figure \Fig{bp1runup}.
\item Task 4. The optional demonstration of convergence was not performed.
\end{itemize}

\begin{figure}[ht]
\hfil\includegraphics[width=2.8in]{bp1/frame0001fig2.png}\hfil
\hfil\includegraphics[width=2.8in]{bp1/frame0003fig2.png}\hfil
\vskip 5pt
\hfil\includegraphics[width=2.8in]{bp1/frame0005fig2.png}\hfil
\hfil\includegraphics[width=2.8in]{bp1/frame0007fig2.png}\hfil
\caption{\label{fig:bp1frames} 
Profile plots for the times specified in Task 2.  For each pair of plots at a particular time, the top frame provides a full view of the incoming wave and the bottom frame provides an expanded view of the inundation area. In some regions the analytic and GeoClaw solutions lie atop one another.}
\end{figure}

\begin{figure}[ht]
\hfil\includegraphics[width=2.8in]{bp1/plotgauge2.png}\hfil
\hfil\includegraphics[width=2.8in]{bp1/plotgauge1.png}\hfil
\caption{\label{fig:bp1gauges} 
Left column: Water level time series at location $x/d = 9.95$.
Right column: Water level time series at location $x/d = 0.25$.
 }
\end{figure}

\begin{figure}[ht]
\hfil\includegraphics[width=5.5in]{bp1/runup.png}\hfil
\caption{\label{fig:bp1runup} 
Runup on canonical beach as a function of time}
\end{figure}

\subsubsection{Lessons learned}

\begin{itemize}
\item This benchmark problem and a good test of the shallow water wave computation against an analytic solution in one dimension.
\item Because of its complexity, the analytical solution should be provided in a data file on the benchmark problem website to ensure all participants are solving the same problem.
\end{itemize}
