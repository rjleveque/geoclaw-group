
\section{Further remarks and suggestions}

\subsection{Comments on the current benchmark problems}

The current set of benchmark problems do a good job of testing some aspects
of a tsunami simulation code.  However, there are some shortcomings
that have become apparent to us in the course of working through these
problems and that could be addressed in the future.

\begin{itemize}
\item Several of the problems are not well specified in terms of the data
provided.  These difficulties
have been noted in our discussion of the individual problems.

\item In some problems there is not a clear description of how the
simulation is supposed to be set up, or how the accuracy of the solution
should be quantified.  Allowing flexibility is perhaps necessary to allow
for differences in capabilities of existing simulation codes, but we feel
this could be better constrained.  In particular, there is no indication in
the problem descriptions of what grid resolution should be used.  There are
requirements to ``demonstrate convergence'', but for practical applications
it is important to know that adequately accurate results can be obtained
on grids with a reasonable resolution in terms of computing time
constraints.  

\item Currently 
there is no requirement to report CPU time required to solve each problem. 
This would be interesting information to have when comparing different
approaches, and would be a necessary component if the benchmark did require
a particular grid resolution, since grid resolution alone is not necessarily
a good indication of computational effort needed.

\item Some of the benchmark problems concern comparison to exact solutions
of the linear or nonlinear shallow water equations.  For these problems any
code that solves the equations in question should converge to the correct
solution, but it may also be of interest to know how rapidly the error goes to
zero, and how good the solution is on under-resolved grids that may be more
representative of what would be used in actual tsunami simulations.

\item Other benchmark problems require comparison with wave tank
experiments.  In some cases (e.g. with breaking waves)
it can not be expected that the code converges to
the experimental results since the equations used in tsunami modeling are
only approximations.  Different codes may use different approximations and
so this comparison may be valuable, but since many codes use the same
shallow water approximations, 
for these problems it would be valuable to have some
agreement as to what a ``converged solution'' of the shallow water equations
looks like. 

\item Benchmark problem \#8a studied in \Sec{bp8a} does not seem to scale
well as a model of a real landslide, and has difficulties
associated with the vertical face that are not likely to be seen in real
landslides, where momentum transfer is probably secondary to the vertical
displacement of the water column in creating a tsunami.  The short
wavelength waves generated by the discontinuity in this problem also
accentuate the need to use dispersive corrections in order to obtain
reasonable approximations.  While dispersive terms may be very 
important for some submarine landslide generated tsunamis, there may be
other cases where it they are less important and the ability to model such
events with shallow water equations is important since these equations can
be solved with explicit methods that are often orders of magnitude faster
than implicit dispersive solvers.  (This may be particularly important in
doing probabilistic hazard assessment requiring a large number of
scenarios.)  We believe it would be valuable to develop landslide benchmarks
that model events such as a large mass failure on the continental slope,
which the current benchmarks do not address.


\item The Okushiri Island benchmark problem \#9 requires comparison to field
observations.  Beyond the technical
difficulties with datasets for this problem, there
are also questions regarding (a) the accuracy of the earthquake source being
used, (b) the accuracy of some of the field observations and tide gauges.
This makes it difficult to assess the accuracy of a simulation code.  This
will always be a problem in comparing with actual events, but our feeling is
that to form a meaningful benchmark there should be some agreement in the
community regarding how large the deviation between the computed solutions
and the observations are expected to be, rather than an expectation that
results converge to observations as the grid is refined.
\end{itemize} 

\subsection{Suggestions for future benchmark problems}

We believe there are other possible benchmark problems that should be
considered by community in order to better test tsunami simulation codes.

\begin{enumerate}
\item The one-dimensional test problems currently involve exact
solutions that are themselves difficult to calculate numerically, e.g.
requiring numerical quadrature of Bessel functions. It is  very
useful that tabulated values of these solutions have been provided.
However, rather than using limited tests for which such ``exact''
solutions are known, it might be preferable to carefully test a 1d
numerical model and show that it converges, and then use this with
very fine grids to generate  reference solutions. 
Fully converged solutions could be provided in tabulated form and could be
as accurate as needed.
It would then be possible to generate a much wider variety of test problems.
In particular, more realistic 
bathymetry could be used, for example on the scale of the
ocean, a continental shelf and beach, rather than modeling only a beach.  

\item High-accuracy one-dimensional reference solutions can also be used to
test a full two dimensional code, by creating bathymetry that varies in only
one direction at some angle to the two-dimensional grid.  A plane wave
approaching such a planar beach would ideally remain one-dimensional, but at
an angle to the grid this would test the two-dimensional inundation
algorithms.

\item This idea 
can be extended to consider radially symmetric problems, such as a
radially symmetric ocean with a Gaussian initial perturbation at the center.
The waves generated should reach the shore at the same time in all
directions, but the shore will be at different angles to the grid in
different locations and it is valuable to compare the accuracy in
different locations. The two-dimensional equations can be reformulated as a
one-dimensional equation in the radial direction (with geometric source
terms) and a very fine grid
solution to this problem can be used as a reference solution.

Features could also be added at one point along the shore and this location
rotated to test the ability of the code to give orientation-independent
results.  Some GeoClaw results of this nature are presented in 
\cite{BergerGeorgeLeVequeMandli:awr11,
LeVequeGeorgeBerger:an11}.

\item A very simple exact solution is known for water in a parabolic bowl,
in which the water surface is linear at all times but the water sloshes
around in a circular motion. This is a good test of wetting and drying
as well as conservation. See for example
\cite{ga-pa-ca:wb07,Thacker:1981} and the test problem in GeoClaw:\\
\url{http://www.clawpack.org/clawpack-4.x/apps/tsunami/bowl-slosh/README.html}


\item Extensive observations are available for recent events such as Chile
2010 or Tohoku 2011, including DART buoys, tide gauges, and field
observations of inundation and runup.  It would be valuable to develop new
benchmark problems based on specific data sets, including specified
bathymetry and earthquake source (or seafloor displacement). 

\end{enumerate} 

