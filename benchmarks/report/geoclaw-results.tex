\documentclass[11pt]{article}

\usepackage{graphicx}

\usepackage{amsmath,amsfonts,amssymb}

\usepackage{color}
\usepackage[colorlinks=true,linkcolor=blue,citecolor=black,filecolor=blue,urlcolor=blue]{hyperref}
\newcommand{\todo}[1]{{\bf \color{blue} [#1]}}
\newcommand{\alert}[1]{{\bf \color{red} [#1]}}
\newcommand{\newsection}{\clearpage\newpage}

\setlength{\textwidth}{6.2in}
\setlength{\oddsidemargin}{0.3in}
\setlength{\evensidemargin}{0in}
\setlength{\textheight}{8.7in}
\setlength{\voffset}{-.7in}
\setlength{\headsep}{26pt}

\input{RJLmacros}

\graphicspath{{./}}


% indicate which bp*.tex files to include: (comment out to include all)
\includeonly{bp7}


\begin{document}

\begin{center}
{{\Large\bf
GeoClaw Results for \\
NTHMP Tsunami Benchmark Problems}
\vskip 5pt
Version of \today}
\end{center}

% NTHMP Model validation workshop : draft reporting template

\vskip 15pt
\noindent
{\bf Team:}  GeoClaw Tsunami Modeling Group, University of Washington

\vskip 5pt
\noindent
{\bf Geographic area:}  The west coast, with a focus on Washington State,
including Puget Sound, the San Juan Island group, and other islands in the
Strait of Juan de Fuca.

\vskip 5pt
\section{Introduction}

(Describe your NTHMP funded effort regarding inundation mapping, i.e., which
area this covers, which types of tsunami sources are affecting your area and
hence which benchmark problems in the list you will be addressing and why.)

Funding is expected from NTHMP partner FEMA’s Risk Mapping, Assessment, and
Planning (Risk MAP) Program, as part of the BakerAECOM coastal engineering
effort for the FEMA California Coastal Analysis and Mapping (CCAMP) Project.
Funding will also be sought to provide Washington State with tsunami
modeling and mapping in support of tsunami hazard assessment and emergency
management planning and education.  Tsunami sources in these geographic
areas include earthquakes and landslides, and we will therefore address
benchmark problems that deal with these sources.

\section{Model description}
(Describe and give relevant references to the model(s) you are using for
your tsunami source, propagation, and coastal impact/inundation simulations.
As a minimum, provide general information on the type of models you are
using, including related restriction/assumptions. Refer to (preferably peer
reviewed) papers where your model equations and other relevant validation
and/or case studies for tsunami or otherwise have been published.)

\section{Benchmark results}
(For each selected (PMEL) benchmark problem, please document how you defined
your model input data, grid and various other parameters to simulate each
problem. 

Provide results as figures (and raw data) following future formatting
instructions, to the relevant site and/or coordinator. Also summarize your
results as simple error indicators of the performance of the model (e.g.,
mean, max, rms errors).)

PMEL-135 refers to \cite{SynolakisBernard:pmel135}.

\newsection
\subsection{BP 1: Single wave on a simple beach (Analytic)}

{\bf Documentation:} A description of this benchmark problem is provided by \cite{bp-description} and \cite{SynolakisBernard:pmel135}.

\subsubsection{Problem Description}
The focus is on comparing computed and analytic solutions for a wave incident on a simple beach, in which:
\begin{itemize}
\item The bathymetry consists of a deep region of constant depth $d$
connected to a sloping beach of angle $\beta = \text{arccot}(19.85)$. 
Note that the toe of the beach is located at $x = X_0 = d \text{cot} \beta$
\item The initial waveform of the wave is given by 
\begin{equation}
\eta(x,0) = H \text{sech}^2(\gamma (x - X_1)/d)
\end{equation}
where $L = \text{arccosh}(\sqrt(20))/\gamma$, $X_1 = X_0 + L$, 
and $\gamma = \sqrt(3H/4d)$. The speed of the wave is given by the following: 
\begin{equation}
u(x,0)=-\sqrt{g/d}\eta(x,0)
\end{equation}
\end{itemize}

\begin{figure}[ht]
\hfil\includegraphics[width=5.0in]{bp1/bp1domain}\hfil
\caption{\label{fig:bp1domain} 
Sketch of canonical beach and approaching wave.
 }
\end{figure}

\subsubsection{Tasks}
\begin{itemize}
\todo{Paul: Please reproduce here the task list in Dmitry's description.}

\end{itemize}

\subsubsection{Problems encountered}

\begin{itemize}
\item The analytic solution of the wave equation was hard to determine and compute.  The analytic solution was obtained from the benchmark problem champion; it would be very helpful if it were provided in an Excel file as part of the benchmark problem description.  
\item No analytical solution was provided for time $t = 25s$
\item The Clawpack code does not currently include maximum runup calculations. An additional module had to be written. 
\end{itemize}

\subsubsection{What we did}

\begin{itemize}
\item Used $g=1$ and no friction.
\item The problem was solved on an $800\times 2$ grid, where the x domain spanned x = -10 to 60.
\item Variable time stepping was allowed, based on a CFL number of 0.9
\end{itemize} 

\subsubsection{Results}
\begin{itemize}
\item Task 1. Good agreement between computed and analytic water level profiles at $t = 35(d/g)^{1/2}$, $t = 45(d/g)^{1/2}$, $t = 55(d/g)^{1/2}$, $t = 65(d/g)^{1/2}$ is presented in Figure \Fig{bp1frames}.  Data were missing from file {\tt canonical\_profiles.txt} for $t=25(d/g)^{1/2}$, so this time was omitted.
\item Task 2. Good agreement between computed and analytic water levels at locations $x/d = 0.25$ and $x/d = 9.95$ during the propagation and reflection of the wave is presented in Figure \Fig{bp1gauges}.
\item Task 3. Maximum runup on the beach was 0.085, as presented in the time series of runup values in Figure \Fig{bp1runup}.
\item Task 4. The optional demonstration of convergence was not performed.
\end{itemize}

\begin{figure}[ht]
\hfil\includegraphics[width=2.8in]{bp1/frame0001fig2.png}\hfil
\hfil\includegraphics[width=2.8in]{bp1/frame0003fig2.png}\hfil
\vskip 5pt
\hfil\includegraphics[width=2.8in]{bp1/frame0005fig2.png}\hfil
\hfil\includegraphics[width=2.8in]{bp1/frame0007fig2.png}\hfil
\caption{\label{fig:bp1frames} 
Profile plots for the times specified in Task 2.  For each pair of plots at a particular time, the top frame provides a full view of the incoming wave and the bottom frame provides an expanded view of the inundation area. In some regions the analytic and GeoClaw solutions lie atop one another.}
\end{figure}

\begin{figure}[ht]
\hfil\includegraphics[width=2.8in]{bp1/plotgauge2.png}\hfil
\hfil\includegraphics[width=2.8in]{bp1/plotgauge1.png}\hfil
\caption{\label{fig:bp1gauges} 
Left column: Water level time series at location $x/d = 9.95$.
Right column: Water level time series at location $x/d = 0.25$.
 }
\end{figure}

\begin{figure}[ht]
\hfil\includegraphics[width=5.5in]{bp1/runup.png}\hfil
\caption{\label{fig:bp1runup} 
Runup on canonical beach as a function of time}
\end{figure}

\subsubsection{Lessons learned}

\begin{itemize}
\item This benchmark problem and a good test of the shallow water wave computation against an analytic solution in one dimension.
\item Because of its complexity, the analytical solution should be provided in a data file on the benchmark problem website to ensure all participants are solving the same problem.
\end{itemize}

\newsection

\subsection{BP 2:
 Solitary wave on composite beach (Analytic)}

{\bf Documentation:}

\begin{itemize}
\item PMEL-135, pp 5 \& 30-33 \cite{SynolakisBernard:pmel135}
\item Problem description provided by Dmitry **CITE**
\item Coastal Hydraulics Laboratory Problem Description \cite{CHLBP2}
\end{itemize}

\subsubsection{What we did}
\begin{itemize}
\item We solved the shallow water wave equation in Cartesian coordinates with $g = 9.81$ and no friction.
\item To specify the incoming wave from the left boundary of our computational domain we used the first ten seconds of  measurements taken at Gage 4.  After ten seconds the left boundary switched to be a non-reflecting boundary.  This boundary is selected since the end of our computational domain is not the end of the physical wave tank.  The implementation of these boundary conditions is described in the section on benchmark problem 7. 
\item We solved on a $600 \times 2$ grid with no adaptive mesh refinement.
\item To impose linearization we scaled the incoming wave by $10^{-4}$ to damp out any nonlinear behavior, then scaled up the gage readings by $10^4$ to compare with analytical solution.
\end{itemize}

\subsubsection{Gage comparisons}
For these Gage comparisons we ran our code linearly with friction set to zero. 

The results for cases A, B, and C are shown in figures \Fig{bp2A}, \Fig{bp2B}, \Fig{bp2C} respectively where Gage 11 is placed at the vertical wall.

\begin{figure}[ht]
\hfil\includegraphics[width=2.8in]{bp2/CaseA/gauge0004fig300.png}\hfil
\hfil\includegraphics[width=2.8in]{bp2/CaseA/gauge0005fig300.png}\hfil
\vskip 5pt
\hfil\includegraphics[width=2.8in]{bp2/CaseA/gauge0006fig300.png}\hfil
\hfil\includegraphics[width=2.8in]{bp2/CaseA/gauge0007fig300.png}\hfil
\vskip 5pt
\hfil\includegraphics[width=2.8in]{bp2/CaseA/gauge0008fig300.png}\hfil
\hfil\includegraphics[width=2.8in]{bp2/CaseA/gauge0009fig300.png}\hfil
\vskip 5pt
\hfil\includegraphics[width=2.8in]{bp2/CaseA/gauge0010fig300.png}\hfil
\hfil\includegraphics[width=2.8in]{bp2/CaseA/gauge0011fig300.png}\hfil
\caption{\label{fig:bp2A} Case A }
\end{figure}

\begin{figure}[ht]
\hfil\includegraphics[width=2.8in]{bp2/CaseB/gauge0004fig300.png}\hfil
\hfil\includegraphics[width=2.8in]{bp2/CaseB/gauge0005fig300.png}\hfil
\vskip 5pt
\hfil\includegraphics[width=2.8in]{bp2/CaseB/gauge0006fig300.png}\hfil
\hfil\includegraphics[width=2.8in]{bp2/CaseB/gauge0007fig300.png}\hfil
\vskip 5pt
\hfil\includegraphics[width=2.8in]{bp2/CaseB/gauge0008fig300.png}\hfil
\hfil\includegraphics[width=2.8in]{bp2/CaseB/gauge0009fig300.png}\hfil
\vskip 5pt
\hfil\includegraphics[width=2.8in]{bp2/CaseB/gauge0010fig300.png}\hfil
\hfil\includegraphics[width=2.8in]{bp2/CaseB/gauge0011fig300.png}\hfil
\caption{\label{fig:bp2B} Case B }
\end{figure}

\begin{figure}[ht]
\hfil\includegraphics[width=2.8in]{bp2/CaseC/gauge0004fig300.png}\hfil
\hfil\includegraphics[width=2.8in]{bp2/CaseC/gauge0005fig300.png}\hfil
\vskip 5pt
\hfil\includegraphics[width=2.8in]{bp2/CaseC/gauge0006fig300.png}\hfil
\hfil\includegraphics[width=2.8in]{bp2/CaseC/gauge0007fig300.png}\hfil
\vskip 5pt
\hfil\includegraphics[width=2.8in]{bp2/CaseC/gauge0008fig300.png}\hfil
\hfil\includegraphics[width=2.8in]{bp2/CaseC/gauge0009fig300.png}\hfil
\vskip 5pt
\hfil\includegraphics[width=2.8in]{bp2/CaseC/gauge0010fig300.png}\hfil
\hfil\includegraphics[width=2.8in]{bp2/CaseC/gauge0011fig300.png}\hfil
\caption{\label{fig:bp2C} Case C }
\end{figure}

\subsubsection{Convergence Study}
We performed a test to see how well Clawpack converged to the analytic solution as we increased the number of grid cells in our computational domain.  We found that as the number of grid cells was increased that the computed solution approached the analytic solution.  The convergence plot is shown in figure \Fig{linearConverge}.

\begin{figure}[ht]
\hfil\includegraphics[width=5in]{bp2/linearCompare}\hfil
\caption{\label{fig:linearConverge} Convergence Plot for Gage 4 in Case C }
\end{figure}


\subsubsection{Lessons learned}
In this benchmark problem we found that using the analytic solution at Gage 4 as boundary conditions on a shorter domain, starting at gage 4, provided more accurate results than using the wave maker position and a longer domain to model the entire tank.  It appears that a similar assumption is made in the provided analytic solutions, as they match up nearly perfectly with the lab data for the first ten seconds.  

Overall this benchmark problem is a good test for one dimensional codes.  The benchmark problem specifications could be improved by specifying the computational domain and the specific data source that should be used to model the incoming wave. 


\newsection

\subsection{BP 3:
 Saucer landslide (Laboratory)}

\newsection

\subsection{BP 4:
 Single wave on simple beach (Laboratory)}


{\bf Documentation:}  PMEL-135, pp 5 \& 30-33

\subsubsection{Problems encountered}

\begin{itemize}
\item Comparisons to high amplitude waves do not correlate well. This would be expected, however, because as the shock forms, it is unlikely that the shallow water wave equations are a good model of the behavior.
\item Excell may not be the best format to present problem data in. 
\end{itemize}

\subsubsection{What we did}

\begin{itemize}
\item Used $g=1$ and no friction.
\item The bathymetry consists of a deep plateau of constant depth $d$ connected to a sloping beach of angle $\beta = arccot(19.85)$. Note that the toe of the beach is located at $x = X_0 = d cot \beta$
\item The initial waveform of the wave is given by 
\begin{equation}
\eta(x,0) = H \text{sech}^2(\gamma (x - X_1)/d)
\end{equation}
where $L = \text{arccosh}(\sqrt(20))/\gamma$, $X_1 = X_0 + L$, 
and $\gamma = \sqrt(3H/4d)$. The speed of the wave is given by the following: 
\begin{equation}
u(x,0)=-\sqrt{g/d}\eta(x,0)
\end{equation}
\item Low amplitude case was solved on a $800\times 2$ grid, where the x domain spanned from -10 to 60.
\item High amplitude case was solved on a $1200\times 2$ grid, where the x domain spanned from -10 to 60.
\item We allowed variable time stepping based on a CFL number of 0.9
\item The amplitude of the wave was set to 0.0185 and 0.03 for the low and high amplitude cases respectively. 
\end{itemize} 

\subsubsection{Frame comparisons}
Compare the numerically computed and experimental water level profiles at $t = 30(d/g)^{1/2}$, $t = 40(d/g)^{1/2}$, $t = 50(d/g)^{1/2}$, $t = 60(d/g)^{1/2}$, and $t = 70(d/g)^{1/2}$ for the low amplitude case. And $t = 15(d/g)^{1/2}$, $t = 20(d/g)^{1/2}$, $t = 25(d/g)^{1/2}$, and $t = 30(d/g)^{1/2}$ for the high amplitude case.

See \Fig{bp2framesa} and \Fig{bp2framesb}
\begin{figure}[ht]
\hfil\includegraphics[width=2.8in]{bp4/lab-185/frame0001fig2.png}\hfil
\hfil\includegraphics[width=2.8in]{bp4/lab-185/frame0002fig2.png}\hfil
\vskip 5pt
\hfil\includegraphics[width=2.8in]{bp4/lab-185/frame0003fig2.png}\hfil
\hfil\includegraphics[width=2.8in]{bp4/lab-185/frame0004fig2.png}\hfil
\vskip 5pt
\hfil\includegraphics[width=2.8in]{bp4/lab-185/frame0005fig2.png}\hfil
\caption{\label{fig:bp2framesa} 
Frames of runup for low amplitude case. Top frame: Full view of incoming wave. Bottom Frame: Zoomed view of inundation area.}
\end{figure}

\begin{figure}[ht]
\hfil\includegraphics[width=2.8in]{bp4/lab-3/frame0001fig2.png}\hfil
\hfil\includegraphics[width=2.8in]{bp4/lab-3/frame0002fig2.png}\hfil
\vskip 5pt
\hfil\includegraphics[width=2.8in]{bp4/lab-3/frame0003fig2.png}\hfil
\hfil\includegraphics[width=2.8in]{bp4/lab-3/frame0004fig2.png}\hfil
\caption{\label{fig:bp2framesb} 
Frames of runup for high amplitude case. Top frame: Full view of incoming wave. Bottom Frame: Zoomed view of inundation area.}
\end{figure}

\subsection{Lessons learned}

\todo{This problem was a good physical verification of the framework developed in BP01. The low amplitude case had, as expected, a high degree of concurence with the predicted model. The higher amplitude case had a much lower degree of concurrence. This may be due to the steep shock that forms, requiring more complicated numerical modeling than that provided by the shallow water wave equations.}

\newsection

\subsection{BP 5:
 Solitary wave on composite beach (Laboratory) }


\subsubsection{Problem specification}

\begin{itemize}
\item PMEL-135, pp 6 \& 37-39.

\item Problem description provided by Elena Tolkova at
\cite{bp-description}:\\
\href{https://github.com/rjleveque/nthmp-benchmark-problems/blob/master/BP05-ElenaT-Solitary_wave_on_composite_beach_laboratory/BP5_description.pdf}
{BP05-ElenaT-Solitary\_wave\_on\_composite\_beach\_laboratory/BP5\_description.pdf} 

\item Coastal Hydraulics Laboratory Problem Description\cite{CHLBP2}
\end{itemize}

This is the same problem as in BP 2, but using the nonlinear shallow water
equations and comparing to laboratory data rather than to the analytic
solution of the linear equations.

\subsubsection{What we did}
\begin{itemize}
\item We solved the shallow water wave equation in Cartesian coordinates with $g = 9.81$ and no friction.
\item To specify the incoming wave from the left boundary of our
computational domain we used the first ten seconds of  measurements taken at
Gage 4.  After ten seconds the left boundary switched to be a non-reflecting
boundary.  This boundary is selected since the end of our computational
domain is not the end of the physical wave tank.  The implementation of
these boundary conditions is described in \Sec{bc}.
\item We solved on a $600 \times 2$ grid with no adaptive mesh refinement. 
\end{itemize}

\subsubsection{Gage comparisons}
The results for cases A, B, and C are shown in figures \Fig{bp5A}, \Fig{bp5B}, \Fig{bp5C} respectively where Gage 11 is placed at the vertical wall.

\begin{figure}[ht]
\hfil\includegraphics[width=2.8in]{bp5/CaseA/gauge0004fig300.png}\hfil
\hfil\includegraphics[width=2.8in]{bp5/CaseA/gauge0005fig300.png}\hfil
\vskip 5pt
\hfil\includegraphics[width=2.8in]{bp5/CaseA/gauge0006fig300.png}\hfil
\hfil\includegraphics[width=2.8in]{bp5/CaseA/gauge0007fig300.png}\hfil
\vskip 5pt
\hfil\includegraphics[width=2.8in]{bp5/CaseA/gauge0008fig300.png}\hfil
\hfil\includegraphics[width=2.8in]{bp5/CaseA/gauge0009fig300.png}\hfil
\vskip 5pt
\hfil\includegraphics[width=2.8in]{bp5/CaseA/gauge0010fig300.png}\hfil
\hfil\includegraphics[width=2.8in]{bp5/CaseA/gauge0011fig300.png}\hfil
\caption{\label{fig:bp5A} Case A }
\end{figure}

\begin{figure}[ht]
\hfil\includegraphics[width=2.8in]{bp5/CaseB/gauge0004fig300.png}\hfil
\hfil\includegraphics[width=2.8in]{bp5/CaseB/gauge0005fig300.png}\hfil
\vskip 5pt
\hfil\includegraphics[width=2.8in]{bp5/CaseB/gauge0006fig300.png}\hfil
\hfil\includegraphics[width=2.8in]{bp5/CaseB/gauge0007fig300.png}\hfil
\vskip 5pt
\hfil\includegraphics[width=2.8in]{bp5/CaseB/gauge0008fig300.png}\hfil
\hfil\includegraphics[width=2.8in]{bp5/CaseB/gauge0009fig300.png}\hfil
\vskip 5pt
\hfil\includegraphics[width=2.8in]{bp5/CaseB/gauge0010fig300.png}\hfil
\hfil\includegraphics[width=2.8in]{bp5/CaseB/gauge0011fig300.png}\hfil
\caption{\label{fig:bp5B} Case B }
\end{figure}

\begin{figure}[ht]
\hfil\includegraphics[width=2.8in]{bp5/CaseC/gauge0004fig300.png}\hfil
\hfil\includegraphics[width=2.8in]{bp5/CaseC/gauge0005fig300.png}\hfil
\vskip 5pt
\hfil\includegraphics[width=2.8in]{bp5/CaseC/gauge0006fig300.png}\hfil
\hfil\includegraphics[width=2.8in]{bp5/CaseC/gauge0007fig300.png}\hfil
\vskip 5pt
\hfil\includegraphics[width=2.8in]{bp5/CaseC/gauge0008fig300.png}\hfil
\hfil\includegraphics[width=2.8in]{bp5/CaseC/gauge0009fig300.png}\hfil
\vskip 5pt
\hfil\includegraphics[width=2.8in]{bp5/CaseC/gauge0010fig300.png}\hfil
\hfil\includegraphics[width=2.8in]{bp5/CaseC/gauge0011fig300.png}\hfil
\caption{\label{fig:bp5C} Case C }
\end{figure}

\subsubsection{Convergence Study}
We performed a test to see how well Clawpack converged to the gage measurements as we increased the number of grid cells in our computational domain.  We found that as the number of grid cells was increased that the computed solution converged and had a shock in approximately the same location as in the gage data.  The results shown in figure \Fig{nonLinearConverge}.

\begin{figure}[ht]
\hfil\includegraphics[width=5in]{bp5/nonLinearCompare}\hfil
\caption{\label{fig:nonLinearConverge} Convergence Plot for Gage 4 in Case C }
\end{figure}

\subsubsection{Lessons learned}
In this benchmark problem we found that using the measured data from Gage 4 as boundary conditions on a shorter domain, starting at this Gage, provided more accurate results than using the wave maker position and a longer domain to model the entire tank.  It appears that a similar assumption is made in the provided analytic solutions, as they match up nearly perfectly with the lab data for the first ten seconds.  

Overall this benchmark problem is a good test for one dimensional codes.  
Case C exhibits dispersion in the laboratory results not seen with the
nonlinear shallow water equations.

The benchmark problem specifications could be improved by specifying the computational domain and the specific data source that should be used to model the incoming wave. 

\subsection{BP 6
  Solitary wave on a conical island (Laboratory)}

{\bf Documentation:}
\begin {itemize}

\item \cite{bp-description}
\item \url{http://chl.erdc.usace.army.mil/chl.aspx?p=s&a=Projects;35}
\item PMEL-135, pp 6 \& 39-45 \cite{SynolakisBernard:pmel135}.
\item Other reference, below
\end{itemize} 

\subsubsection {Problems encountered}

\begin {itemize}
\item Details of the laboratory setup and, therefore, the computational domain could not not be determined by the available documentation.  In particular, the following items were not documented: (a) the distance from the wavemaker face to the island center and (b) open gaps at each end of the wavemaker.
\item Erroneous entries were found in data files ts2a.txt, ts2b.txt and ts2cnew1.txt.  Several entries of the letter "M" triggered read-in error messages; they were replaced by linear interpolation or extrapolation of neighboring values.
\end{itemize} 

\subsubsection{What we did}

\begin{itemize}
\item Used $g=9.81$ and no friction.
\item Used the computational domain presented in \Fig{bp6/Domain.pdf}, developed after personal communication with Michael Briggs, U.S. Army Corps of Engineers, who  provided additional information on physical details of the laboratory experiment.
\item Used open boundary conditions for the top, bottom and right walls, and for the gaps between the ends of the wavemaker and the top and bottom walls.
\item Used boundary conditions for the face of the wavemaker at the left wall corresponding to the ...  
\item Simulated Cases A and C, each with three different grid sizes and resolution, to demonstrate convergence:  61 X 47 (50 cm), 121 X 93 (25 cm) and 241 X 185 (12.5 cm)
\end{itemize}

\subsubsection{Results}


{\bf References:}
\begin {itemize}

\item briggs1:  Briggs, M.J., Synolakis, C.E., Harkins, G.S. (1994):  Tsunami runup on a conical island.  Proc. of Waves — Physical and Numerical Modelling, 21-24 August 1994, Vancouver, Canada, V1, 446-455.
\item Briggs, M.J., C.E. Synolakis, G.S. Harkins, and D. Green (1995): Laboratory experiments of tsunami runup on a circular island. Pure Appl. Geophys., 144, 569-593.
\item Briggs, M.J., C.E. Synolakis, G.S. Harkins, and D. R. Green (1996): Runup of solitary waves on a circular island, in Proceedings of the Second International Long-Wave Runup Models, Friday Harbor, Washington, 12-16 September 1995, 363-374.
\item Liu, P.L.-F., Y.-S. Cho, K. Fujima (1994):  Numerical Solutions of Three-Dimensional Run-up on a Circular Island, 21-24 August 1994, Vancouver, Canada, V2, 1031-1040.
\item Liu, P.L.-F., Y.-S. Cho, M.J. Briggs, U. Kânoglu, and C.E. Synolakis (1995): Runup of solitary waves on a circular island. J. Fluid Mech., 302, 259-285.
\item Yeh, H., P. Liu, M. Briggs, C. Synolakis, (1994): Propagation and amplification of tsunamis at coastal boundaries, Nature 372, 353-355.
\end{itemize}

\begin{figure}[ht]
\hfil\includegraphics[width=1.5in]{bp6/Domain.pdf}\hfil
\end{figure}

\todo{Move these to references.bib}


\subsection{BP 7:
 Monai valley beach (Laboratory)}

{\bf Documentation:}  PMEL-135, pp 6 \& 45-46.

\subsubsection{Problems encountered}

\begin{itemize}
\item The input data only goes out to 20 seconds.

\item The first waves are modeled well but later waves are not seen in the
computation.  This is perhaps related to first point.  

\item Data is not provided for run-up in the valley.
\end{itemize}

\subsubsection{What we did}

\begin{itemize}
\item Used $g=9.81$ and no friction.
\item We used the given initial wave to specify a boundary condition at the left
boundary up to time 20.  This was done by filling ghost cells each time step
at the left edge of the computational domain, with cell centers at $x = -
\frac 1 2 \Delta x$ and $x = -\frac 3 2 \Delta x$.
The depth at any such point $x_0$ at arbitrary time $t_0$ should be roughly
the same as the depth at $x=0$ at time $t_0 - x_0/c_0$, where $c_0 =
\sqrt{gh_0}$ is the wave speed based on the constant depth.  This value was
estimated by 
interpolating from the given time trace at $x=0$.
To fully specify the ghost cell values we also need the velocity or momentum
in these cells.  This is estimated by assuming the wave is a right-going
simple wave, so that the Riemann invariant $u - 2\sqrt{gh}$ is constant
through the wave.  Inserting the interpolated $h$ value and then setting this
equal to $-2\sqrt{gh_0}$ gives the expression for the velocity $u$ used in
the ghost cell.

\item After time 20, switched to non-reflecting boundary condition 
at left boundary, so reflected waves exit.  
(Note that the wave tank was much longer than computational domain specified.)
\item We solved on $423\times 243$ grid (same as bathymetry)
\item We also solved on $160\times 100$ grid (coarser grid, 
not aligned with bathymetry) for comparison.
\end{itemize} 

\subsubsection{Gauge comparisons}

See \Fig{bp7gauges}.

\begin{figure}[ht]
\hfil\includegraphics[width=2.8in]{bp7/figs423/gauge0005fig300.png}\hfil
\hfil\includegraphics[width=2.8in]{bp7/figs160/gauge0005fig300.png}\hfil
\vskip 5pt
\hfil\includegraphics[width=2.8in]{bp7/figs423/gauge0007fig300.png}\hfil
\hfil\includegraphics[width=2.8in]{bp7/figs160/gauge0007fig300.png}\hfil
\vskip 5pt
\hfil\includegraphics[width=2.8in]{bp7/figs423/gauge0009fig300.png}\hfil
\hfil\includegraphics[width=2.8in]{bp7/figs160/gauge0009fig300.png}\hfil
\caption{\label{fig:bp7gauges} 
Left column: on $423\times 243$ grid (same as given bathymetry).
Right column: $160\times 100$ grid (coarser and not aligned with
bathymetry).
  }
\end{figure}



\subsubsection{Frame comparisons}

See \Fig{bp7framesA} and \Fig{bp7framesB} for comparisons of the Frames 10,
25, 40, 55, and 70 from the overhead movie with GeoClaw results at roughly
corresponding times.  These results are from the $423\times 243$ grid (same
as given bathymetry).

The movie had a rate of 30 fps, so the frames are 0.5 seconds apart. However,
it is not clear what the starting time was for Frame 1 relative to the
simulation time.   In the Benchmark Description \cite{bp-description}, it is
stated that ``frame 10 approximately occurs at 15.3 seconds'', but then later
``it is recommend that each modeler find times of the snapshots that best fit
the data''.   We found reasonably good agreement starting at 15 seconds for
Frame 1 and then taking 0.5 second increments, as shown in \Fig{bp7framesA}
and \Fig{bp7framesB}.

The yellow dashed lines on the frames from the movie show the approximate
shoreline (and were provided as part of the benchmark specification
\cite{bp-description}).  The actual shoreline location is of course somewhat
ambiguous in the movie, and also in the computation.  The figures of the
GeoClaw computation show the shoreline two different ways:
\begin{itemize}
\item The cells colored blue are finite volume cells where the fluid depth is
greater than 0.0001 m. Those colored green have less fluid or are dry.
\item The black dashed line is a contour line where depth $= 0.002$ m, which
agrees better with the movie frames and might be a depth that can actually be
detected in the movie frames.
\end{itemize} 

\begin{figure}[ht]
\hfil\includegraphics[width=1.5in]{bp7/movie/Frame10.png}\hfil
\hfil\includegraphics[width=2.8in]{bp7/figs423/frame0005fig10.png}\hfil
\vskip 5pt
\hfil\includegraphics[width=1.5in]{bp7/movie/Frame25.png}\hfil
\hfil\includegraphics[width=2.8in]{bp7/figs423/frame0007fig10.png}\hfil
\vskip 5pt
\hfil\includegraphics[width=1.5in]{bp7/movie/Frame40.png}\hfil
\hfil\includegraphics[width=2.8in]{bp7/figs423/frame0009fig10.png}\hfil
\caption{\label{fig:bp7framesA} 
Left column: Frames from the movie.
Right column: Zoomed view of computation.
  }
\end{figure}


\begin{figure}[ht]
\hfil\includegraphics[width=1.5in]{bp7/movie/Frame55.png}\hfil
\hfil\includegraphics[width=2.8in]{bp7/figs423/frame0011fig10.png}\hfil
\vskip 5pt
\hfil\includegraphics[width=1.5in]{bp7/movie/Frame70.png}\hfil
\hfil\includegraphics[width=2.8in]{bp7/figs423/frame0013fig10.png}\hfil
\vskip 5pt
\caption{\label{fig:bp7framesB} 
Left column: Frames from the movie.
Right column: Zoomed view of computation.
  }
\end{figure}


\subsection{Lessons learned}

\begin{itemize}
\item This problem has data that is fairly well specified, and has wave tank
geometry that scales up to a reasonable physical tsunami problem (since it
was designed by scaling down a physical problem).  

\item Solutions to the shallow water equations fit the data quite well, as
found both in our experiments and by other modellers.  This gives a
reassuring test of the validity of shallow water equations for real
tsunamis.

\end{itemize} 

\newsection

\subsection{BP 8:
 3D landslide (Laboratory)}

There are plans to replace this benchmark problem with a new one.

The old benchmark problem is in bp8a.tex.

\newsection

\subsection{BP 9:
 Okushiri island (Field)}

{\bf Documentation:}  
\begin{itemize}
\item PMEL-135, pp 8 \& 48-53 \cite{SynolakisBernard:pmel135}.

\item A problem description is also provided by Frank Gonz\'alez 
\href{https://github.com/rjleveque/nthmp-benchmark-problems/blob/master/BP09-FrankG-Okushiri_island/Description.pdf}
{BP09-FrankG-Okushiri\_island/Description.pdf} 
at \cite{bp-description}.  
\end{itemize} 

\newsection

\subsection{BP 10:
 PNG landslide-generated tsunami (Field)}

\newsection

\subsection{BP 11:
 Landslide (Lab)}



\bibliographystyle{plain}
\bibliography{references}
\end{document}

