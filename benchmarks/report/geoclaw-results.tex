\documentclass[11pt]{article}

\usepackage{graphicx}

\usepackage{amsmath,amsfonts,amssymb}

\usepackage{color}
\usepackage[colorlinks=true,linkcolor=blue,citecolor=black,filecolor=blue,urlcolor=blue]{hyperref}
\newcommand{\todo}[1]{{\bf \color{blue} [#1]}}
\newcommand{\alert}[1]{{\bf \color{red} [#1]}}

\setlength{\textwidth}{6.2in}
\setlength{\oddsidemargin}{0.3in}
\setlength{\evensidemargin}{0in}
\setlength{\textheight}{8.7in}
\setlength{\voffset}{-.7in}
\setlength{\headsep}{26pt}

\input{RJLmacros}

\graphicspath{{./}}

\begin{document}

\begin{center}
{{\Large\bf
GeoClaw Results for \\
NTHMP Tsunami Benchmark Problems}
\vskip 5pt
Version of \today}
\end{center}

% NTHMP Model validation workshop : draft reporting template

\vskip 15pt
\noindent
{\bf Team:}  GeoClaw Tsunami Modeling Group, University of Washington

\vskip 5pt
\noindent
{\bf Geographic area:}  The west coast, with a focus on Washington State,
including Puget Sound, the San Juan Island group, and other islands in the
Strait of Juan de Fuca.

\vskip 5pt
\section{Introduction}

(Describe your NTHMP funded effort regarding inundation mapping, i.e., which
area this covers, which types of tsunami sources are affecting your area and
hence which benchmark problems in the list you will be addressing and why.)

Funding is expected from NTHMP partner FEMA’s Risk Mapping, Assessment, and
Planning (Risk MAP) Program, as part of the BakerAECOM coastal engineering
effort for the FEMA California Coastal Analysis and Mapping (CCAMP) Project.
Funding will also be sought to provide Washington State with tsunami
modeling and mapping in support of tsunami hazard assessment and emergency
management planning and education.  Tsunami sources in these geographic
areas include earthquakes and landslides, and we will therefore address
benchmark problems that deal with these sources.

\section{Model description}
(Describe and give relevant references to the model(s) you are using for
your tsunami source, propagation, and coastal impact/inundation simulations.
As a minimum, provide general information on the type of models you are
using, including related restriction/assumptions. Refer to (preferably peer
reviewed) papers where your model equations and other relevant validation
and/or case studies for tsunami or otherwise have been published.)

\section{Benchmark results}
(For each selected (PMEL) benchmark problem, please document how you defined
your model input data, grid and various other parameters to simulate each
problem. 

Provide results as figures (and raw data) following future formatting
instructions, to the relevant site and/or coordinator. Also summarize your
results as simple error indicators of the performance of the model (e.g.,
mean, max, rms errors).)

PMEL-135 refers to \cite{SynolakisBernard:pmel135}.

\subsection{BP 1:
  Single wave on a simple beach (Analytic)}

{\bf Documentation:}  PMEL-135, pp 5 \& 18-30.

\subsection{BP 2:
 Solitary wave on composite beach (Analytic)}

{\bf Documentation:}  PMEL-135, pp 5 \& 30-33

\subsection{BP 3:
 Saucer landslide (Laboratory)}

\subsection{BP 4:
 Single wave on simple beach (Laboratory)}

{\bf Documentation:}  PMEL-135, pp 6 \& 34-37.

\subsection{BP 5:
 Solitary wave on composite beach (Laboratory) }

{\bf Documentation:}  PMEL-135, pp 6 \& 37-39.

\subsection{BP 6:
 Solitary wave on a conical island (Laboratory)}

{\bf Documentation:}  PMEL-135, pp 6 \& 39-45

\newpage
\subsection{BP 7:
 Monai valley beach (Laboratory)}

{\bf Documentation:}  PMEL-135, pp 6 \& 45-46.

\subsubsection{Problems encountered}

\begin{itemize}
\item The input data only goes out to 20 seconds.

\item The first waves are modeled well but later waves are not seen in the
computation.  This is perhaps related to first point.  

\item Data is not provided for run-up in the valley.
\end{itemize}

\subsubsection{What we did}

\begin{itemize}
\item Used $g=9.81$ and no friction.
\item Used given initial wave to specify a boundary condition at the left
boundary up to time 20.
\item After time 20, switched to non-reflecting at left boundary, so
reflected waves exit.  
(Wave tank was much longer than computational domain specified.)
\item Solved on $423\times 243$ grid (same as bathymetry)
\item Solved on $160\times 100$ grid (not aligned with bathymetry)
\end{itemize} 

\subsubsection{Gauge comparisons}

See \Fig{bp7gauges}.

\begin{figure}[ht]
\hfil\includegraphics[width=2.8in]{bp7/figs423/gauge0001fig300.png}\hfil
\hfil\includegraphics[width=2.8in]{bp7/figs160/gauge0001fig300.png}\hfil
\vskip 5pt
\hfil\includegraphics[width=2.8in]{bp7/figs423/gauge0002fig300.png}\hfil
\hfil\includegraphics[width=2.8in]{bp7/figs160/gauge0002fig300.png}\hfil
\vskip 5pt
\hfil\includegraphics[width=2.8in]{bp7/figs423/gauge0003fig300.png}\hfil
\hfil\includegraphics[width=2.8in]{bp7/figs160/gauge0003fig300.png}\hfil
\caption{\label{fig:bp7gauges} 
Left column: on $423\times 243$ grid (same as given bathymetry).
Right column: $160\times 100$ grid (coarser and not aligned with
bathymetry).
  }
\end{figure}


\clearpage
\newpage
\subsection{BP 8:
 3D landslide (Laboratory)}

\subsection{BP 9:
 Okushiri island (Field)}

{\bf Documentation:}  PMEL-135, pp 8 \& 48-53.

\subsection{BP 10:
 PNG landslide-generated tsunami (Field)}

\subsection{BP 11:
 Landslide (Lab)}

\section{Lessons learned}

(Make any relevant comments or observation regarding the benchmark itself,
the available data provided, the relevance of the benchmark to tsunami
science in general and to validating your specific model in particular. 
Report problems and make recommendations regarding improving the benchmark
or its data, as appropriate.)



\bibliographystyle{plain}
\bibliography{references}
\end{document}

